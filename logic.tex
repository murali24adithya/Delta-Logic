\section{Delta-Logics}
\label{sec:delta-logics}

In this section, we define delta-logics extending many-sorted first-order logic with least fixpoints
and background axiomatizations of some of the sorts.

Let us fix a many-sorted first-order signature $\Sigma = (S, {\cal F}, {\cal P}, {\cal C}, {\cal G}, {\cal R})$ where 
$S=\{\sigma_0, \ldots, \sigma_n\}$ is a nonempty finite set of sorts, 
${\cal F}$, ${\cal P}$, and ${\cal C}$ are sets of function symbols, relation symbols,
and constant symbols, respectively, and
${\cal G}$ and ${\cal R}$ are function and relation symbols that will be recursively defined. 
These symbols have implicity defined an appropriate arity and a type signature.


Let $\sigma_0$ be a special sort that we refer to as the \emph{location} sort, which
will model locations on the heap. The other sorts, which we refer to as background sorts, can be arbitrary and
constrained to conform to some theory (such as a theory of arithmetic or a theory of sets).

We assume the following restrictions:
\begin{itemize}
	\item We assume all functions in ${\cal F}$ map either from tuples of one sort to itself or from the foreground
	 sort $\sigma_0$ to a background sort $\sigma_i$. Relations in ${\cal P}$ are over tuples of one sort only.
	\item The functions in ${\cal F}$ whose domain is over the foreground sort $\sigma_0$ are \emph{unary}.
	      Also, relations over the foreground sort $\sigma_0$ are unary relations.
	\item Recursively defined functions (in ${\cal G}$) are all unary functions from the foreground sort $\sigma_0$ 
	         to the foreground sort or a background sort. Recursively defined relations (in ${\cal R}$) are all unary relations on the foreground sort $\sigma_0$.
\end{itemize}

The restriction to have unary functions from the foreground sort (which models locations) is sufficient
to model pointers on the heap (unary functions from $\sigma_0$ to $\sigma_0$) 
and to model data stored in the heap (like the key stored at locations modeled as a function from $\sigma_0$ to a
background sort of integers).
This restriction will greatly simplify the presentation of delta-logics below.
The restriction of having unary recursively defined functions and relations will also simplify the notation.
Note that recursive definitions such as $\textit{lseg}(x,y)$ that are binary can be written recursively
as unary relations such as $\textit{lseg}_y(x)$ (i.e., parameterized over the variable $y$) with recursion
on the variable $x$. 

The logic FO+\textit{lfp} that we use consists of a set of recursive definitions of (unary) predicates and functions (with least
fixpoint semantics) and a quantifier-free formula that uses these definitions. For a simpler exposition, a definition of a recursive function $R$ will be of the form:
$R ::=_{lfp} \varphi(x, R(p_1(x)),..R(p_n(x)))$
%Don't we have to account the possibility of mutual recursion, or at least mention that mutual recursion is OK?
where $p_i$ are unary functions and $\varphi{}$ is monotonic, i.e, the $R(p_i(x))$ occur in the defintion under an even number of negations. 

The latttice that we have for computing the least fixpoints for predicates is $\{\top{}, \bot{} \}$ with $\bot{} < \top{}$. For functions whose range is a domain $\mathcal{D}$, this lattice is $\mathcal{D} \cup{} \{\bot{}\}$, where $\bot{} < d$ for every $d$ in $\mathcal{D}$. The semantics of atomic formulas is that they evaluate to \textit{false} whenever they involve a term involving $\bot{}$, when the formula is under an even number of negations; and to \textit{true} otherwise.

Furthermore, we restrict the kind of recursively defined predicates to have definitions of the form $R(x) ::= \varphi(x, e)$ where $e$ is a set of conjunctions involving $R(p_i(x))$. 
%We also disallow the compositional application of multiple functions defined on the foreground sort (modelling pointer fields); and %hence do not have terms of the form $p_1(p_2(x))$. This is not really a restriction, since a recursive definition with such terms can %be rewritten into multiple recursive definitions, each one satisfying our requirements. 
This restricts definitions to have a unique dependence on their successors, and most natural definitions for heaps satisfy this property. Similarly, the restriction on recursively defined functions is that they have definitions of the form $G(x) ::= t(x, e)$ for some term $t$, such that $e$ is a term involving terms of the form $G(p_i(x))$.

We parameterize delta-logics by a finite set of first-order variables 
$\Delta = \{ v_1, \ldots v_n\}$.
Delta-logic formulas are Boolean combinations of \emph{contextual} formulae and \emph{delta-specific} formulae.
We now define the former.

\paragraph{Contextual Formulae}
Intuitively, a contextual formula $\varphi(\vec{x})$ is a formula that evaluates
on a model $M$ while \emph{ignoring} the functions/relations on the locations interpreted
for the variables in $\Delta$. 

More precisely, a semantic definition of the contextual logic over $\Sigma$ with respect 
to $\Delta$ is as below. First, we shall define when a pair of models over the same
universe and interpretation 'differ only on $\Delta$'.

\begin{definition}[Models differing only on $\Delta$]
Let $M$ and $M'$  be two $\Sigma$-models with universe $U$ that interpret constants the same way,
and let $I$ be an interpretation of variables over $U$.
Then we say $(M, I)$ and $(M', I)$ differ only on $\Delta$ if:
\begin{itemize}
	\item  for every
	function symbol $f$ and for every $l \in U$, $\llbracket f \rrbracket_M(l)  \not = \llbracket f \rrbracket_{M'}(l)$ only if
	there exists some $v \in \Delta$ such that $I(v) = l$.
	
	\item  for every relation symbol $S$ and for every $l \in U$, $\llbracket S \rrbracket_M(l)  \not \equiv \llbracket S \rrbracket_{M'}(l) $ only if 
there exists some $v \in \Delta$ such that $I(v) = l$.\qed	
\end{itemize}
\end{definition}

Intuitively, the above says that the interpretation of the (unary) functions and relations of the two models are 
precisely the same for all elements in the universe that are not interpretations of the variables in $\Delta$.

An FOL+lfp formula over $\Sigma$, $\varphi(\vec{x})$, is a contextual formula
if the formula does not distinguish between models that differ only on $\Delta$:


\begin{definition}[Contextual formulae]
An FOL+lfp formula over $\Sigma$, $\varphi(\vec{x})$, is a contextual formula
if for every two $\Sigma$-models $M$ and $M'$ with the same universe and every interpretation $I$ 
such that $(M, I)$ and $(M', I)$ differ only on $\Delta$, 
$M, I \models \varphi$ iff $M', I \models \varphi$.\qed
\end{definition}

Contextual formulae can be easily written using syntactic restrictions where the formulae (and the recursive definitions)
are written so that every occurrence of $f(t)$ (where $f$ is a function symbol) or $P(t)$ (where $P$ is a relation symbol),
where $t$ is a term of type $\sigma_0$, 
is guarded by the clause ``$t \not \in \Delta$'', which is short for $\bigwedge_{v \in \Delta} t \not = v)$.

\noindent Let us now give some examples of contextual formulae.

\paragraph{Example:}
\emph{
Let us fix a finite set $\Delta$ of first-order variables.\\
The recursive definition $$\textit{ls}^*(x) :=_{\textit{lfp}} (x=\textit{nil} \vee (x \not = nil \wedge x \not \in \Delta \wedge \textit{ls}^*(n(x))) 
 \vee (x \not = nil \wedge \bigvee_{v \in \Delta} (x\!=\!v \wedge b_v)))$$ 
 is a contextual formula/definition with respect to $\Delta$.
The above defines lists but by ``imbibing'' facts about whether a location $v$ in $\Delta$ is a list using the
free Boolean variable $b_v$. These Boolean variables play the role of the interface variables/parameters mentioned earlier.
%The definition says that $x$ points to a list if it either is equal to the constant $\textit{nil}$,
%or is not equal to $\textit{nil}$ and either $x$ is not in $\Delta$ and $n(x)$ is a list or $x$ is in $\Delta$, and the corresponding
%Boolean variable holds. 
The least fixpoint semantics of the above definition gives a unique definition: $ls(u)$ is true iff
there is a path using the pointer $n()$ that either ends in $\textit{nil}$ or ends in a node $v$ in $\Delta$ where $b_v$ is true. 
Note that the formula $n(x)$ is guarded by the check $x \in \Delta$, and hence is a contextual formula.
Changing the model to reinterpret $n()$ over $\Delta$ (but preserving the interpretation of the variables $b_v$, $v \in \Delta$) 
will not change the definition of $\textit{ls}$.
}

\paragraph{Delta-specific formulae:}
A delta-specific formula is a quantifier-free formula where every occurrence of a term of the form $f(t)$, (for an uninterpreted
function $f$) $t \in \Delta$. Furthermore the formula does not refer to any of the recursively defined functions/predicates.

\paragraph{Delta-logic formulae:}
A delta-logic formula is a Boolean combination of contextual formulae and delta specific-formulae.

\emph{Example: The formula $u'\!\!=\!\!n(u) \wedge \textit{ls}^*(u')$ is a delta-logic formula that uses the above definition of $ls$.}






