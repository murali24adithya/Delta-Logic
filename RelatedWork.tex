\section{Related Work}
There is rich literature on decidable fragments of separation logic \cite{Reynolds2002} with inductive predicates. While the first works such as  \cite{BerdineCalcagnoO'Hearn2004, BerdineCalcagnoO'Hearn2005, CookHaaseChristoph2011,PerezAntonioRybalchenko2011 } handled only list segments, there has been further improvement. The work in \cite{PerezRybalchenko2013} handles only list segments, but allows conjunctions with pure formulae from arbitrary decidable SMT theories. The work in \cite{BrotherstonFuhsPerez2014} argues decidablility for separation logic with arbitrary/user-defined inductive predicates, and \cite{LeTatsutaSun2017,LeSunChin2016 } adapts the idea of using equisatisfiable `base' formulas instead of inductive predicates to include arithmetic constraints satisfying a certain property. However, none of these would be directly applicable to program verification since they do not address entailment, and do not contain an implication in their logic.  The work in\cite{PerezRybalchenko2013} does contain implications, but it does not allow for spatial formulas other than points-to, list segment and in particular not provide support for multiple natural measures which our work provides. However, in the light of our broader work on context-logics and delta-logics, it is valuable to consider the satisfiability problem in the context-logic since it models the unchanging heap and does not need to address entailments, of which these works above provide useful decision procedures.

In constrast, works such as \cite{PiskacWiesZufferey2014, PiskacWiesZufferey2014Tool} do address program verification and provide a decidable logic on spatial formulas that when reduced to SMT can be extended with other decidable SMT theories. There is reasonable overlap in the spirit of this claim to our work, where allowing expressions in arbitrary decidable SMT theories on $\Delta{}$  would preserve decidability as well, and would fit well within the paradigm of delta-logics. But where \cite{PiskacWiesZufferey2014} can support trees/tree segments, it does not provide support for measures. Works such as\cite{MadhusudanParlatoQiu2011, IosifRogalewiczSimacek2013} differ similarly. %\textbf{maybe add more? Where to I place \cite{BouajjaniDruagoiEnea2009} ?}.\\
Lastly there is work in the spirit of \cite{ChinDavidNguyen2007, QiuGargStefanescu2013}, where one dispenses with decidability and is therefore incomparable in that respect with our work. In particular, \cite{QiuGargStefanescu2013} is a powerful method capable of verifying all of our examples, but such logics are incomplete and cannot give meaningful counterexamples like our work does.