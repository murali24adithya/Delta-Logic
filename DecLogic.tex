\section{A Decidable Delta Logic on Lists with List Measures}
\label{sec:declogic}
In this section, we will define a delta-logic on linked lists equipped with \emph{list measures}--- measures of
list segments that include its length, heaplet, the multiset of keys stored in the list (say, in a data-field {\tt key}),
and the minimum and maximum keys stored in it. We prove that the quantifier-free first-order logic fragment of this delta-logic 
is \emph{decidable}. 

We prove decidability by first proving that the corresponding contextual logic formuale can be translated to equisatisfiable quantifier-free first-order
formulae. Consequently, a delta-logic formula, being a Boolean combination of contextual formulae and delta-specific
formulae can be translated to quantifier-free formulae as well, which is decidable using a Nelson-Oppen combination of decidable theories of arithmetic, sets, and uninterpreted functions.

%The translation in fact will allow us to decide delta-logic formulas of the form $\alpha \wedge \beta$ obtained as verification %conditions
%in the section above, where $\alpha$ is a delta-logic formula on lists and list measures and $\beta$ is over a decidable
%Nelson-Oppen combinable quantifier-free theories, and where $\alpha$ and $\beta$ share Boolean and first-order variables.
%At the end of the section, we outline some generalizations of our result.

\subsection*{The contextual logic of list measures}
We  now define a contextual logic over list segments and measures over them. Notice from Section~\ref{sec:separability} that the separation of formulae into delta-logic introduces recursive definitions in the contextual logic (parameterized over various sets of variables) and additionally a rank function for each such definition. However, it is easy to see that for the definitions of lists and measures, all the rank functions (over a given set of parameters) coincide, since the existence of a meaningful value for each of these measures is predicated upon the referred location pointing to a list. This motivates the following recursive definitions for our contextual logic of lists and measures.

As usual, let us fix a set of first-order variables $\Delta$.
Let us also fix a single pointer field $n$.

\begin{definition}[Recursive Definitions for the Logic of List Measures (LM)]
Let us fix a set of \emph{parameter variables} $P$ that consists of the following of sets of variables:
a set of Boolean variables $LS_z^v$, a set of variables with type set of locations
$HLS_z^v$, a set of variables of type multiset of keys $MSKEYS_z^v$, and
a set of variables of type integer (type of keys in general) $Max_z^v$ and $Min_z^v$, where $z,v$ range over $\Delta$.

The contextual logic of list measures (LM) wrt $\Delta$ and parameter variables $P$ is defined using the following recursive definitions, which depend crucially on the parameter variables $P$.
\begin{itemize}
	\item We have unary relations $ls_z^P$ that capture linked list segments 
	       that end in $z$, where
	         the relation for a location $v$ in $\Delta$ is imbibed using the Boolean variables $LS_z^v$ ($v \in \Delta$),
	      and where $z$ is any element of $\Delta$ or the constant location $\textit{nil}$.
	      (The relation $ls_\textit{nil}()$ captures whether a location points to a list
	       ending with $\textit{nil}$.)
	
	This is defined as follows:
	$$\textit{ls}_z^P(x)\!\!:=_{\textit{lfp}}\!\! \left( x\!\!=\!\!\textit{z} \vee \left( x \!\not =\! z \wedge x \!\not=\! nil \wedge x \not \in \Delta \wedge \textit{ls}_z^P(n(x))\right) 
	\vee\right.$$
	$$~~~~ \left.\left(x \!\not =\! z \wedge x \in \Delta \wedge \bigwedge_{v \in \Delta} (x=v \Rightarrow LS_z^v)\right)\right)$$ 
	           

    \item We have recursive definitions that capture the heaplet of such list-segments, where the heaplet
of list-segments from an element $v$ in $\Delta$ to $z$ (where $z \in \Delta \cup \{nil\}$) is imbibed from the set variable $\HLS_z^v$:
\begin{align*}
\textit{hls}_z^P(x) & :=_{\textit{lfp}} & \emptyset & \textit{~~if~} \sem{x}\!\!=\!\!\sem{z} \\
& & \{x\} \cup \textit{hls}_z^P(n(x)) & \textit{~~if~} \sem{x} \!\not =\! \sem{z} \wedge \sem{x} \!\not=\! \sem{nil} \wedge \sem{x} \not \in \sem{\Delta} \\
& & \HLS\,_z^v & \textit{~~if~} \sem{x} \not = \sem{z} \wedge \sem{x}=\sem{v} \wedge v \in \Delta 
\end{align*}

\item We have similar recursive definitions that capture the multiset of data elements stored in list segments, the maximum/minimum element stored inthe list, and a predicate capturing sortedness of list segments. We omit these definitions: see Appendix for detailed definitions.    

\end{itemize}
\end{definition}

%The above definitions can be written in usual syntax using $\textit{ite}$ expressions; we omit this 
%formulation.

We define the \emph{contextual logic of list-measures (LM)} to be quantifier-free formulae that use only the recursive 
definitions of $LM$ mentioned above, and combine them as described in Figure~1.
The logic $LM$ allows first-order variables that range over locations, keys, and integers. Note that
dereferencing pointers of locations is completely disallowed--- the delta-specific formula will refer to such dereferences, depending on the particular state modeled.
Here $\Delta$ is assumed to be a subset of the free location variables in the contextual formula.
Formulae in $LM$ are allowed to refer to recursive predicates/functions defined using over various sets of parameter variables.


\begin{figure}
   \begin{align*}
   \textit{Location} ~~\textit{Term} ~~lt & ::=  x \mid p_i(y) \mid \texttt{nil} \hfill{\textit{~~~~~~~~~~~~~where~} y \not \in \Delta}\\
   \textit{Integer} ~~\textit{Term} ~~it & ::=  c \mid len_z^P(lt) \mid it+it \\
   \textit{Key} ~~\textit{Term} ~~keyt & ::=  c \mid \textit{key}(lt) \mid max_z^P(lt) \mid min_z^P(lt) \mid keyt+keyt \\
   \textit{Heaplet}~~ \textit{Term} ~~hlt  & ::=  \emptyset \mid \{ lt \} \mid hls_z^P(lt) \mid hlt \cup hlt \mid hlt \cap hlt \mid hlt \setminus hlt\\
   \textit{MultisetKeys}~~ \textit{Term} ~~mskt & ::= \emptyset \mid mskeys_z^P(x) \mid mskt \cup_m mskt \mid mskt \cap_m mskt \mid mskt \setminus_m mskt\\
   ~&~\\
   \textit{Formulas}~~ \varphi &  ::=  true \mid false \mid ls_z^P(x) \mid sorted_z^P(x) \mid lt = lt \mid lt \in hlt \mid hlt \subseteq hlt \mid hlt = \emptyset \mid \\
   ~& ~~~~~ it < it \mid it = it \mid keyt < keyt \mid keyt = keyt \mid \\
   ~& ~~~~~ keyt \in mskt \mid mskt \subseteq_m mskt \mid \varphi \vee \varphi \mid \neg \varphi
   \end{align*}
 \caption{The context logic $LM$ of list measures involving list-segments, heaplets, multisets of keys, max, min, and sortedness.} 
\end{figure}

\subsection{Translating contextual formulae to quantifier-free recursion-free formulae}
We can now state the main result of this section:

\begin{theorem}
\label{decidableLMbig}
	For any quantifier-free formula $\varphi(\mathcal{P}, X)$ of $LM[ls, hls, rank, len, $\\$min, max, sorted]$, the quantifier-free and recursion-free
	formula $\psi(\mathcal{P}, X)$ obtained $\varphi$ obtained from the translation below
	satisfies the following property: for any interpretation of the free variables
	in $\mathcal{P} \cup X$, there is a model for $\varphi$ iff there is a model for $\psi$.
\end{theorem}


\begin{corollary}
	The delta-logic of linear measures is decidable.
\end{corollary}

Let us fix a set of sets of parameters $\mathcal{P} = \{P_1, \ldots P_k\}$
(we encourage the reader to fix $k=2$ in their mind while reading the section, as it's the
most common and the logic VCs translate to, as shown in Section~\ref{sec:VC2Delta}).

We will first describe the decision procedure and its proof of correctness for the fragment
of $LM$ that involves only the three recursive definitions $ls_z^P$, $hls_z^P$, and $rank_z^P$,
where $P \in \{\mathcal{P}\}$, which we will refer to as $LM[ls,hls,rank]$. 
Then we will extend the procedure to handle the logic with all the other measures; this latter
proof requires more expressive decision procedures and pseudo-measures that make its proof harder.

Let us assume a quantifier-free $LM[ls,hls,rank]$ formula $\varphi$ which is a $\Delta$-logic formula
w.r.t a finite set of variables $\Delta$.
Assume the set of (free) location variables occurring in $\varphi$ is $X = \{x_1, \ldots x_n\}$ with
$\Delta \subseteq X$. 

In order to determine whether there is a model satsifying $\varphi$, we need to construct a universe
of locations, an interpretation of the variables in $X$, and the heap
(with the single pointer field $n()$) on all locations \emph{outside} $\Delta$ (the definition
of $n()$ on $\Delta$, by definition, does not matter).

Our decision procedure relies intuitively on the following observations. First, note that the locations
reached by using the $n()$ pointer any number of times forms the relevant set of locations that $\varphi$'s
truth can depend on (as $\varphi$ is quantifier-free and has recursive definitions that only use the $n$-pointer).
There are three distinct cases to consider when pursuing the paths using the $n$-pointer on a location $x$:
(a) the path may reach a node in $\Delta$, (b) the path may reach a node that is reachable also from another location in $X$,
or (c) the path may never reach a location in $\Delta$ nor a location that is reachable from another location in $X$. 

The key idea is to \emph{collapse} paths where the reachability of those locations from locations in (interpreted by) $X$ does not change.
More precisely, let $L$ be the set of all locations reachable from $X$ such that $l$ is in $\Delta$ or 
for every location $l'$ reachable from $X$ such that $n(l')=l$, the set of nodes in $X$ that have a path
to $l'$ is different from the set of nodes in $X$ that have a path to $l$.
%Consider writing 'either in X or are locations of earliest points of intersection of n()-paths from locations in X' or some clearer variant

It is easy to see that there are at most $|X|-1$ locations of the above kind that are distinct from $\Delta$,
since the paths can merge at most $|X|-1$ times forming a tree-like structure. Our key idea is now to 
represent these list segments that connect these kinds of locations \emph{symbolically}, summarising the 
measures on these list segments. Since there are only a bounded number of such locations and hence list
segments, we can compute recursive definitions of linear measures involving them using quantifier-free
and recursive-definition-free formulae.

We construct a formula $\psi$ that is satisfiable iff $\varphi$ is satisfiable, as follows.
First, we fix a new set (distinct from $X$) of location variables $V={v_1, \ldots v_{|X|-1}}$,
to stand for the merging locations described above.
We introduce an uninterpreted function $T: V \cup (X\setminus \Delta) \longrightarrow V \cup X \cup \{\bot\}$
($\bot$ is used to signify that the n()-path on the location never intersects $X \cup{} L$).
Let $Z$ be the set of variables in $X$ such that the recursive definitions $ls_z^P$, $hls_z^P$, $rank_z^P$, for some
$P \in \mathcal{P}$, occur in $\varphi$.

\medskip
\noindent 
{\bf $\psi$ is the conjunct of the following formulae:}
\begin{itemize}
	\item The formula $\varphi$ (but with recursive definitions treated as uninterpreted relations and functions).
	\item For every $z \in Z$, we introduce an uninterpreted function $Dist_z: V \cup (X \setminus \Delta) \longrightarrow \mathbb{N} \cup \{\bot\}$ that
             is meant to capture the distance from any location in $V \cup X$ to $z$, if $z$ is reachable from
               that location without going through $\Delta$, and is $\bot$ otherwise. We add the constraint:
          $$\wedge_{v \in V\cup (X\setminus \Delta)} \big[ (Dist_z(v)\!=\!0 \Leftrightarrow v=z) ~\wedge ~~~~~~~~~~~~~~~~~~~~~~~~~~~~~~~~~~~~~~~~~~~~~~~~~~~~~~$$
          $$ v \!\not =\! z \Rightarrow \big(~\left( (T(v)\!=\!\bot \vee Dist_z(T(v))\!=\!\bot) \Rightarrow Dist_z(v) \!=\! \bot
                 \right)~~~~~~~~~~~~~~~$$
          $$  ~~~~~  \left. \wedge ~((T(v) \not = \bot \wedge Dist_z(T(v)) \not = \bot) \Rightarrow Dist_z(v) = Dist_z(T(v))+1
                \big) ~\right)\big] $$
              
	\item For every $x \in X$, and for every $P \in \mathcal{P}$, we have a conjunct:
		     $$ls_z^P(x) \Leftrightarrow (Dist_z(x)\not= \bot \vee \bigvee_{v \in \Delta} (Dist_v(x) \not = \bot
		       \wedge LS_z^v))$$
		     
    \item For every $x \in X$, $z \in Z$, and for every $P \in \mathcal{P}$, we have a conjunct:
             $$ \left( Dist_z(x) \!=\! \bot \Rightarrow rank_z^P\!(x)\!=\!\bot\right) ~~\wedge~~ 
               \left( Dist_z(x) \!\not =\! \bot \Rightarrow rank_z^P\!(x)\!=\!RANK_z^P \right)$$
               
    \item We capture the heaplets of list-segments from $v \in V \cup (X \setminus \Delta)$ to $T(v)$ (excluding both
              end-points) using a set
             of locations $H(v)$
             and constrain them so that they are pairwise disjoint and do not contain the locations $X$:
             $$ \bigwedge_{x \in X, v \in V \cup (X \setminus \Delta)} x \not \in H(v) ~~\wedge 
             \bigwedge_{v,v' \in V \cup (X \setminus \Delta)} (v \!\not =\! v' \Rightarrow H(v) \cap H(v') = \emptyset)$$  
    
    \item We can then precisely capture the heaplet $hls_z^P(x)$ by taking the union of all heaplets of
           list segments lying on its path to $z$. We do this using the following constraint, for each
           $v \in X \cup V$:
           $$(Dist_z(v)=\bot \Rightarrow hls_z^P (v) = \emptyset) \wedge (hls_z^P(z)= \emptyset)\wedge $$
           $$(Dist_z(v) \not = \bot \wedge v \not = z) \Rightarrow hls_z(v)= H(v) \cup \{v\} \cup hls_z(T(v))$$

\end{itemize}

Note that the formula $\psi$ is quantifier-free and over the combined theory of arithmetic, uninterpreted functions, and sets.

We can show the correctness of the above translation:
%\begin{theorem}
%	For any quantifier-free formula $\varphi(\mathcal{P}, X)$ of $LM[ls,hls,rank]$, the quantifier-free and recursion-free
%	 formula $\psi(\mathcal{P}, X)$ obtained $\varphi$ obtained from the translation above
%	 satisfies the following property: for any interpretation of the free variables
 %   in $\mathcal{P} \cup X$, there is a model for $\varphi$ iff there is a model for $\psi$.
%\end{theorem}
\begin{theorem}
The statement of Theorem~\ref{decidableLMbig} holds for the fragment $LM[ls,hls,rank]$.
\end{theorem}

We now turn to the more complex logic $LM[ls, hls, rank, len, mskeys, min, $\\$max, sorted]$, and show that
any quantifier-free formula $\varphi$ in the logic can be similarly translated. First, we model the multiset
of keys, minimum and maximum values and sortedness of each list-segment from $v$ to $T(v)$ 
(where $v \in (X \setminus \Delta) \cup V$),
which is outside $\Delta$, using multiset variables $mskeys\mu(v)$, integer variables $min\mu(v)$, $max\mu(v)$,
and $len\mu(v)$ and boolean variables $sorted\mu(v)$.
We can also aggregate them, as above, to express the sets $mskeys_z(x)$, $min_z(x)$, $max_z(x)$, $len_z(x)$
and $sorted_z(x)$, for each $z \in Z$ and
each $x \in (X \setminus \Delta) \cup V$, similar to definitions of $hls_z(x)$ as defined above.
A point of note is that the recursive definition of sortedness across segments is expressed
by using both $Min_z(x)$ and $Max_z(x)$ definitions, though the recursive definition of sortedness uses 
only minimum--- this is needed as expressing when the concatenation of sorted list segments is sorted
requires the max value of the first segment. We skip these definitions as they are easy to derive.

The main problem that remains is in \emph{constraining} these measures so that they can be the measures
of the \emph{same} list segment, i.e, be the attributes not of a pseudo-model. The following constraints capture this, for each 
$v \in (X \setminus \Delta) \cup V$:
\begin{itemize}
	\item The cardinality of $hls\mu(v)$ must be $len\mu(v)$.
	\item The cardinality of $mskeys\mu(v)$ must be $len\mu(v)$.
	\item $min\mu(v)$ and $max\mu(v)$ must be the minimum and maximum elements of $mskeys\mu(v)$.
	\item If $min\mu(v)=max\mu(v) \not = \bot$, then $sorted\mu(v)$ can only be true.
\end{itemize}

The intuition is that any measures meeting the above constraints can be realized using true list
segments. As for the fourth clause above, notice that any list segment with minumum element different
from maximum can be realized by either a sorted or an unsorted list.

The above constraints on measures, though seemingly simple, are hard to shoehorn into existing decidable theories
(though the fourth constraint can be easily expressed).
The first two constraints can be expressed using quantifier-free BAPA~\cite{BAPA} (Boolean Algebra with Presburger
Arithmetic) constraints, which is decidable. We can get around defining the minimum of list segments
by having the set of keys store only offsets from the minimum (and including the key $0$ always). However,
capturing max and sortedness measures as well while preserving decidability seems hard.

Consequently, we give a new decision procedure that exploits the setup we have here.
First, note that we can restrict the formulae that use sets containing keys to involve
only membership testing of free variables in them, combinations using union and intersection,
and checking emptiness of derived sets. We can \emph{disallow checking non-emptiness} as non-emptiness
of a set $S$ can always be captured by demanding $k \in S$, for a fresh free variable $k$.

%without affecting satisfiability.

Our primary observation is that we can then restrict the multiset of keys to be over a \emph{bounded}
universe of elements. This bounded universe consists of one element for each free variable of type key 
in the formula (call this $K$), and, in addition, will consist of one element for each Venn region formed by the multiset
of keys for each segment $(v,T(v))$ of the context's heap. The idea of introducing an element for each Venn
region is not new, and is found in many works that deal with combinations of sets and cardinality constraints~\cite{BAPA}.

%One can show that if there is a satisfiable model, then by collapsing all elements that are different
%than $K$ and that are in the same Venn region can be identified by the unique representative we have chosen
%for that region. Taking unions and intersections will not affect the multiplicities of elements in the Venn
%region and the formula cannot distinguish between them (as it is quantifier-free).

\begin{figure}[t]
\begin{tabular}{| l| c| r| l|r|} 
\hline
&&&&\\
\textbf{Correct programs} & \textbf{\#VCs} & \textbf{time(s)} & \textbf{Buggy programs} & \textbf{time(s)} \\ 
&&&&\\
\hline
append(x: list, y: list) & 4 & 57~~~ & buggy\_append(x: list, y: list) & 0.2 \\ 
copyall(x: list) & 4 & 183~~~ & buggy\_copyall(x: list) & 36 \\ 
detect\_cycle(x: list) & 6 & 5~~~ & buggy\_detect\_cycle(x: list) & 0.7 \\ 
deleteall(x: list, k: key) & 5 & 10~~~ & buggy\_deleteall(x: list, k: key) & 0.1 \\ 
find(x: list, k: key) & 3 & 6~~~ & buggy\_find(x: list, k: key) & 0.1 \\ 
insert(x: list, k: key) & 4 & 38~~~ & buggy\_insert(x: list, k: key & 0.2\\ 
insert\_front(x: list, k: key) & 1 & 3~~~ & buggy\_insert\_front(x: list, k: key) & 0.1\\ 
insert\_back(x: list, k: key) & 4 & 20~~~ & buggy\_insert\_back(x: list, k: key) & 0.2\\ 
reverse(x: list) & 3 & 14~~~ & buggy\_reverse(x: list) & 0.1\\ 
\hline
sorted\_append(x: list, y: list) & 4 & 54~~~ & buggy\_sorted\_append(x: list, y: list) & 0.2 \\ 
sorted\_deleteall(x: list, k: key) & 5 & 2~~~ & buggy\_sorted\_deleteall(x: list, k: key) & 1.1 \\ 
sorted\_insert(x: list, k: key) & 4 & 17~~~ & buggy\_sorted\_insert(x: list, k: key) & 1.2 \\ 
sorted\_reverse(x: list) & 3 & 37~~~ & buggy\_sorted\_reverse(x: list) & 0.5 \\ 
sorted\_merge(x: list, y: list) & 8 & 300~~~ & buggy\_sorted\_merge(x: list, y: list) & 0.7 \\ 
\hline
%\caption{Decision procedure for LM: experimental results} % needs to go inside longtable environment
%\label{experiments}
\end{tabular}
\caption{Experimental results for the decision procedure for the delta-logic LM}
\end{figure}


Once we have bounded the universe of keys, we can represent a multiset of keys using a set of natural numbers
that represent the multiplicity of elements, and write the effect of unions and intersections
using Presburger arithmetic. The cardinality of the multiset is the sum of these numbers
and the minimum and maximum key can be expressed using the smallest and largest keys in the finite universe
with multiplicity greater than $0$. 
We can hence translate the formula into a quantifier-free formula, 
giving the required theorem.
This proves Theorem~\ref{decidableLMbig}.

Note that the above procedure introduces an exponential number of variables, and hence poses challenges
to be effective in practice. There are several possible ways of mitigating this. 
First, there is existing work~(see ~\cite{BAPA2}) on reasoning with BAPA that argues and builds practical 
algorithms that introduce far smaller universes in practice. Second, in the case where we allow only
combinations of sets using union (and not intersection), and allow checking subset constraints and emptiness,
we can show that introducing a \emph{single} new element in the universe other than $K$ suffices. 
The reason
is that without intersections, the identity of the elements do not matter and their multiplicities
are preserved by representing with only one element. We exploit this in our implementation below. 