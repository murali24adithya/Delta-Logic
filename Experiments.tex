\subsection{Implementation and evaluation of decision procedure for LM}
\label{sec:experiments}

\begin{sloppypar}We implemented the decision procedure for the delta-logic $LM[ls, hls, rank, len, mskeys, min, max, sorted]$ using the reduction to SMT described above, solved using Z3. We applied our technique to a suite of list-manipulating programs.\end{sloppypar}

The specifications for the programs were strong. For example, programs such as \texttt{deleteall} which removed a certain key from a list were also verified with the additional requirement any other arbitrary key had to be preserved in multiplicity across the program. We could also verify such properties as the heaplet of the resulting list being a subset of the original heaplet. The \texttt{detect\_cycle} program was able to express the existence of cycles using the predicates in $LM$, and used them to verify Floyd's tortoise and the hare algorithm.

The results are summarized in Figure~\ref{exp_table}. The left column shows results of verifying correct prorgrams, while the column on the right shows results on `buggy' variants of these programs. The buggy programs were obtained by expressing weaker annotations. Our tool worked well on all these examples and for the buggy programs, the satisfying valuation it provided was sufficiently informative to diagnose the error. To the best of our knowledge, ours is the only decidable verification tool that can handle this suite of examples.

The experiements were performed on a machine with an Intel\textsuperscript{\textregistered}  Core\texttrademark  i7-7600U processor with clock speeds upto 3.9GHz.