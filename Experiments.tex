\subsection{Implementation and evaluation of decision procedure for LM}

\begin{sloppypar}We implemented the decision procedure for the delta-logic $LM[ls, hls, rank, len, mskeys, min, max, sorted]$ using the reduction to SMT described above, solved using Z3. We applied our technique to a suite of list-manipulating programs.\end{sloppypar}

The specifications for the programs were strong. For example, the program \textit{sorted\_merge} merges two disjoint sorted lists pointed to by $x$ and $y$, and has the precondition they are indeed so, the postcondition asserting that the result is a sorted list whose heaplet is the union of the heaplets of lists pointed to by $x$ and $y$, and similarly for the multiset of keys. It is also possible to express that for an arbitrary node in either list, the key was unaltered.

The results are summarized in the table below. The left column shows results of verifying correct prorgrams, while the column on the right shows results on `buggy' variants of these programs. The buggy programs were obtained by expressing weaker annotations. Our tool worked well on all these examples and for the buggy programs, the satisfying valuation it provided was sufficiently informative to diagnose the error. To the best of our knowledge, ours is the only decidable verification tool that can handle this suite of examples.

The experiements were performed on a machine with an Intel\textsuperscript{\textregistered}  Core\texttrademark  i7-7600U processor with clock speeds upto 3.9GHz.