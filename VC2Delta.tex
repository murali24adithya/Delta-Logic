	\section{Translating VCs to Delta Logic}
Let us consider the Hoare Triple: ($\alpha_{\textit{pre}}$, $T$, $\alpha_{\textit{post}}$) such that $\alpha_{pre}$ and $\alpha_{post}$ are FO+\textit{lfp} formulae. The program manipulates pointers and data fields, which we model using unary functions. We can then write $\alpha_{\textit{pre}}$ over sets of variables, fields and recursively defined functions $X, P$ and $R$ respectively and similarly $\alpha_{\textit{post}}$ over sets $X', P'$ and $R'$ such that for every $x \in{} X$, there is a corresponding $x' \in{} X'$ abd similarly for $f \in{} P$, there is a corresponding symbol $f' \in{} P'$. Intuitively, this is used to identify the values of program variables and distinguish the state of the pointer/data fields in the universe after the execution of the program. Consequently, for every recursively defined function $r \in{} R$, there is a corresponding $r' \in{} R'$ such that $r' = r[P'/P]$ is a substitution of $f'$ for the corresponding $f$ in the definition of $r$. \\\

We can also write $T$, the formula describing the program transformation, over $X \cup{} X' \cup{} X_{tmp}$ and $P \cup{} P'$ such that $\Delta{} \subseteq{} X \cup{} X' \cup{} X_{tmp} $, where $T = T_1 \land{} T_2$ can be written as a conjuction of:\\
\begin{itemize}
\item A quantifier-free formula $T_1$ such that any subterm of the form $f(t)$ for some $f \in{} P \cup{} P'$ and some term $t$ must have $t = v$ for some $v \in{} \Delta{}$ (this is to ensure that the pointer and data fields are only referenced at variables in $\Delta{}$), and\\
\item A formula $T_2$: $\bigwedge\limits_{f \in{} P} \left(\forall{}z.\,z\notin{}\Delta{} \implies{} f'(z) = f(z) \right)$ (this describes that the pointer and data fields are changed only on variables in $\Delta$). \\
\end{itemize} 
This is possible since the program changes the values of the data and pointer fields on elements only within $\Delta$, and the values of the variables in the state resulting after the program execution can be written as expressions of the values in the state before, with only a finite number of temporary variables.\\\

It is clear that the VC that captures the goven Hoare Triple will be $\alpha_{\textit{pre}} \land{} T \land{} \neg{} \alpha_{\textit{post}}$, which by the above is a formula in FO+\textit{lfp} within our given signature. From Theorem \ref{Separability}, we have that an FO+\textit{lfp} formula can be written equivalently as a delta-logic formula. Therefore, $\alpha_{\textit{pre}}$ and $\alpha_{\textit{post}}$ can be rewritten to equivalent delta-logic.\\
However, from the theorem we will also see in particular that the resursive definitions of functions and predicates in $\alpha_{\textit{post}}$ use functions from $P'$ only on arguments in $\Delta{}^{c}$. Therefore, from $T_2$ we have that these can be replaced with corresponding functions in $P$ since the pointer and data fields of (locations interpreted by) variables not in $\Delta{}$ are unaltered by the program. Then, we also remove $T_2$ since the symbols in $P'$ are no longer referred to anywhere else on an argument not in $\Delta{}$.\\\

Therefore, the VC can be written equivalently as a delta-logic formula (since $T$ does not contain any resursively defined functions as subterms). Since delta-logic formulae are boolean combinations of context-logic fomulae and delta-specific formulae, satisfiability of the VC then becomes the meaningful question of asking independently for a model of a context, a prior state and a resulting state (with a common valuation for finitely many shared first-order variables), such the context when applied over a model of the prior state satisfies the precondition, and applied over a model of the resulting state falsifies the postcondition. 